\documentclass[lineaire_algebra_oplossingen.tex]{subfiles}
\begin{document}

\part{Zelfreflectie 3}
\section{oef 1}
\subsection*{a)}
vrij, lineair onafhankelijk
\subsection*{b)}
niet vrij, lineair afhankelijk
\section{oef 2}
De tweede bewering. Deze oefening beschrijft precies wat er mis is met de cursus.
\section{oef 3}
Ja. Het triviaal voorbeeld is natuurlijk $\{\vec{0}\}$. Als dit niet voldoende is, bestaan er ook nog binaire ringen. Dit zijn ook vectorruimten, als we de bewerkingen goed defini\"eren.
\section{oef 4}
De deelruimten van de gepunte ruimte zijn ruimtes, vlakken of rechten door de oorsprong, en de deelruimte met enkel de oorsprong. Twee lineair onafhankelijke vectoren spannen een vlak op. één vector spant een rechte op. Dire lineair onafhankelijke vectoren spannen opnieuw de ruimte op. 
\section{oef 5}
Een basis is voortbrengend en vrij (definitie). Noem $\beta = \{v_1,v_2,...,v_n\}$ de basis van $V$.
\begin{proof}
$\beta$ is maximaal vrij want elke vector in $V$ in is een lineaire combinatie van de vectoren in $\beta$. Als één van die vectoren aan $\beta$ toegevoegd zou worden, zou $\beta$ dus niet meer vrij zijn. Dit is precies de definitie van maximaal vrij.\\
Stel dat $\beta$ niet minimaal voortbrengend zou zijn zou $\beta$ uitgedunt kunnen worden tot $\beta'$ door er $b$ uit te halen zodat $\beta'$ nog steeds voortbrengend is. Als dit waar is dan zou $b$ een lineaire combinatie zijn van de vectoren uit $\beta'$. Dit is in contradictie met het feit dat $V$ vrij is. $\beta$ is dus minimaal voortbrengend.
\end{proof}
\section{oef 6}
\section{oef 7}
\section{oef 8}
\section{oef 9}
\section{oef 10}
\section{oef 11}
\section{oef 12}
\section{oef 13}
\section{oef 14}
\section{oef 15}
\section{oef 17}
\section{oef 18}
\end{document}