\documentclass[lineaire_algebra_oplossingen.tex]{subfiles}
\begin{document}

\part{Zelfreflectie 3}
\section{oef 1}
\subsection*{a)}
vrij, lineair onafhankelijk
\subsection*{b)}
niet vrij, lineair afhankelijk
\section{oef 2}
De tweede bewering. Deze oefening beschrijft precies wat er mis is met de cursus.
\section{oef 3}
Ja. Het triviaal voorbeeld is natuurlijk $\{\vec{0}\}$. Als dit niet voldoende is, bestaan er ook nog binaire ringen. Dit zijn ook vectorruimten, als we de bewerkingen goed defini\"eren.
\section{oef 4}
\section{oef 5}
\section{oef 6}
\section{oef 7}
\section{oef 8}
\section{oef 9}
\section{oef 10}
\section{oef 11}
\section{oef 12}
\section{oef 13}
\section{oef 14}
\section{oef 15}
\section{oef 17}
\section{oef 18}
\end{document}