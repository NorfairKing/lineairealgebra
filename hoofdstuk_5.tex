\documentclass[lineaire_algebra_oplossingen.tex]{subfiles}
\begin{document}

\newpage
\part{Hoofdstuk 5}
\section{Bewijzen uit het boek}

\subsection{Stelling 5.2 p 177}
Zij $A\in \mathbb{R}^{n\times n}$ een vierkante matrix.
\subsubsection*{Te Bewijzen}
\begin{center}
Een getal $\lambda\in\mathbb{R}$ is een eigenwaarde van $A$.
\end{center}
\[\Leftrightarrow\]
\begin{center}
$\lambda$ is een nulpunt van de karakteristieke veelterm $det(X\mathbb{I}_n - A)$ van $A$
\end{center}
\subsubsection*{Bewijs}
\begin{proof}
Bewijs van een equivalentie.
\begin{itemize}
\item $\Rightarrow$\\
Omdat $\lambda$ een eigenwaarde is van $A$ bestaat er een (eigen)vector $v$ zodat de volgende bewering geldt\footnote{Zie Definitie 5.1 p 177}.
\[
A\cdot v = \lambda v
\]
We weten dat $\lambda v =  \lambda \mathbb{I}_n \cdot v$ en dat de matrixvermenigvuldiging distributief is als ze bepaald is\footnote{Zie Eigenschappen 1.22 b}.
\[
A\cdot v - \lambda \mathbb{I}_n \cdot v = \vec{0} = (A-\lambda\mathbb{I}_n)\cdot v = \vec{0}
\]
Omdat $v$ per definitie geen nulvector is moet de determinant van $(A-\lambda\mathbb{I}_n)$ nul zijn opdat opdat $(A-\lambda\mathbb{I}_n)\cdot v = \vec{0}$ geldt.

\item $\Leftarrow$\\
Als $det(A-\lambda\mathbb{I}_n) = 0$ geldt voor $\lambda$ met $v$ als eigenvector, dan geldt zeker het volgende.
\[
(A-\lambda\mathbb{I}_n)\cdot v = \vec{0}
\]
\end{itemize}
\end{proof}

\subsection{Voorbeeld 5.4 p 178}
\subsubsection*{1)}
We zoeken nog een oplossing van de volgende vergelijking.
\[
\begin{pmatrix}
-1 & -1\\
-1 & -1\\
\end{pmatrix}
\cdot
\begin{pmatrix}
x\\y
\end{pmatrix}
=
\vec{0}
\]
De oplossingsverzameling hiervan is de volgende.
\[
V = \{-\lambda,\lambda|\lambda\in\mathbb{R}\}
\]
Als we nu $\lambda = 1$ kiezen krijgen we als eigenvector bij voorbeeld $(-1,1)$.

\subsubsection*{5)}
Voor elke eigenvector $v$ van $D$ geldt dat de afgeleide van $v$ een veelvoud is van $v$. Elke constante functie is een eigenvector en $0$ is de eigenwaarde voor die functies.


\section{Oefeningen 5.9}
\end{document}