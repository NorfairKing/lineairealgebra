\documentclass[lineaire_algebra_oplossingen.tex]{subfiles}
\begin{document}

\section{Examen Januari 2010}
\subsection{Vraag 1 (Theorie)}
\subsubsection*{(a)}
Dit bewijs staat letterlijk in de cursus. Zie Stelling 3.37 p. 107 (\ref{3.37}).

\subsubsection*{(b)}
Afgezien van het feit dat een algoritme een probleem zou hebben met oneindigdimensionale vectorruimten, zou dit geen probleem mogen vormen.

\subsection{Vraag 2 (Theorie)}
Dit bewijs staat letterlijk in de cursus. Zie Stelling 5.18 p. 190 (\ref{5.18}).
\subsection{Vraag 3}
\subsubsection*{(a)}
\subsubsection*{(b)}
\subsection{Vraag 4}
\subsubsection*{(a)}
\subsubsection*{(b)}
\subsection{Vraag 5}
\subsection{Vraag 6}
\subsection{Vraag 7}

\end{document}