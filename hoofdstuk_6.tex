\documentclass[lineaire_algebra_oplossingen.tex]{subfiles}
\begin{document}

\chapter{Hoofdstuk 6}

\section{Bewijzen uit het boek}

\subsection{Opmerking 6.2 p 222}
Zij $(\mathbb{R},V,+)$ een reële vectorruimte.
\subsubsection*{Te Bewijzen}
Een inproduct is lineair in de tweede component.
\[
\forall w_1,w_2,v\in V, \forall \lambda_1,\lambda_2 \in \mathbb{R}:
\langle v,\lambda_1w_1+\lambda_2w_2\rangle = \lambda_1\langle v,w_1\rangle + \lambda_2\langle v,w_2\rangle
\]
\subsubsection*{Bewijs}
\begin{proof}
Kies drie willekeurige vectoren $w_1,w_2,v\in V$ en beschouw het volgende inproduct. Door lineariteit in de eerste component geldt de gelijkheid.
\[
\langle \lambda_1w_1+\lambda_2w_2,v\rangle = \lambda_1\langle w_1,v\rangle + \lambda_2\langle w_2,v\rangle
\]
Door symmetrie geldt dat deze twee leden ook gelijk zijn aan de volgende.
\[
\langle v,\lambda_1w_1+\lambda_2w_2\rangle = \lambda_1\langle v,w_1\rangle + \lambda_2\langle v,w_2\rangle
\]
\end{proof}

\section{Oefeningen 6.8}






\end{document}