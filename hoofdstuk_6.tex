\documentclass[lineaire_algebra_oplossingen.tex]{subfiles}
\begin{document}

\chapter{Hoofdstuk 6}

\section{Bewijzen uit het boek}

\subsection{Opmerking 6.2 p 222}
Zij $(\mathbb{R},V,+)$ een reële vectorruimte.
\subsubsection*{Te Bewijzen}
Een inproduct is lineair in de tweede component.
\[
\forall w_1,w_2,v\in V, \forall \lambda_1,\lambda_2 \in \mathbb{R}:
\langle v,\lambda_1w_1+\lambda_2w_2\rangle = \lambda_1\langle v,w_1\rangle + \lambda_2\langle v,w_2\rangle
\]
\subsubsection*{Bewijs}
\begin{proof}
Kies drie willekeurige vectoren $w_1,w_2,v\in V$ en beschouw het volgende inproduct. Door lineariteit in de eerste component geldt de gelijkheid.
\[
\langle \lambda_1w_1+\lambda_2w_2,v\rangle = \lambda_1\langle w_1,v\rangle + \lambda_2\langle w_2,v\rangle
\]
Door symmetrie geldt dat deze twee leden ook gelijk zijn aan de volgende.
\[
\langle v,\lambda_1w_1+\lambda_2w_2\rangle = \lambda_1\langle v,w_1\rangle + \lambda_2\langle v,w_2\rangle
\]
\end{proof}

\subsection{Voorbeeld 6.6 p 224}
Definieer $\langle f,g\rangle$ als volgt.
\[
\langle f,g\rangle = \int_a^bf(x)\cdot g(x)dx
\]
Beschouw de vectorruimte $(\mathbb{R},C[a,b],+)$.
\subsubsection*{Te Bewijzen}
$\langle f,g\rangle$ is een inproduct.
\subsubsection*{Bewijs}
\begin{proof}
\begin{itemize}
\item $\langle f,g\rangle$ is lineair in de eerste component.
\[
\forall v_1,v_2,w\in V, \forall \lambda_1,\lambda_2 \in \mathbb{R}:
\langle \lambda_1v_1+\lambda_2v_2,w\rangle = \int_a^b\lambda_1 v_1(x)\cdot \lambda_2 v_2(x)dx
\]
\[
= \lambda_1\int_a^b v_1(x)\cdot \lambda_2\int_a^b v_2(x)dx = \lambda_1\langle v_1,w\rangle + \lambda_2\langle v_2,w\rangle
\]

\item $\langle f,g\rangle$ is symmetrisch.
\[
\forall v,w\in V: \langle v,w\rangle = \int_a^bv(x)\cdot w(x)dx= \int_a^bw(x)\cdot v(x)dx = \langle w,v\rangle
\]

\item $\langle f,g\rangle$ is positief.
\[
\forall v\in V: \langle v,v\rangle = \int_a^bv(x)\cdot v(x)dx \ge 0
\]
Dit is positief als want het is een kwadraat, namelijk van $\int_a^bv(x)dx$

\item $\langle f,g\rangle$ is definiet.
\[
\forall v\in V: \langle v,0\rangle = \vec{0}
\]
\[
\Leftrightarrow \int_a^bv(x)\vec{0}(x) dx = 0
\]
\end{itemize}
\end{proof}

\subsection{Stelling 6.11 p 228}
Zij $(\mathbb{R},V,+,\langle\cdot,\cdot\rangle)$ een inproductruimte met bijhorende norm.
\subsubsection*{Te Bewijzen}
\begin{enumerate}
\item 
\[
\forall \lambda\in\mathbb{R}, v\in V: \Vert\lambda v\Vert = \vert\lambda\vert\cdot \Vert v\Vert
\]

\item
\[
\forall v\in V: \Vert v\Vert\ge 0
\]

\item
\[
\forall v\in V: \Vert v\Vert\ge 0 \Leftrightarrow v=\vec{0}
\]
\end{enumerate}
\subsubsection*{Bewijs}
\begin{proof}
Direct bewijs.\\
Kies een willekeurige $v \in V$ en een $\lambda \in \mathbb{R}$
\begin{enumerate}
\item
\[
\Vert\lambda v\Vert = \sqrt{\langle \lambda v,\lambda v\rangle} = 
\sqrt{\lambda^2 \langle v,v\rangle}= \lambda\sqrt{\langle v,v\rangle} = \vert\lambda\vert\cdot \Vert v\Vert
\]

\item
\[
\Vert v\Vert= \sqrt{\langle v,v\rangle} \ge 0
\]
Een wortel is steeds positief en een inproduct is positief dus de wortel kan steeds getrokken worden \footnote{Zie Definitie 6.1 p 222}0.

\item
\item $\Rightarrow$\\
\[
\Vert v\Vert = 0 \Leftrightarrow \sqrt{\langle v,v\rangle}=0
\]
Dit kan enkel waar zijn als hetgeen onder de wortel ook nul is. Dit is zo omdat het inproduct definiet is \footnote{Zie Definitie 6.1 p 222}. 
\end{enumerate}
\end{proof}

\subsection{Stelling 6.14 p 229}
Zij $(\mathbb{R},V,+,\langle\cdot,\cdot\rangle)$ een inproductruimte.
\subsubsection*{Te Bewijzen}
\begin{enumerate}
\item 
\[
\forall v,w \in V: \vert\langle v,w\rangle\vert \le \Vert v\Vert\cdot \Vert w\Vert
\]
\item
\[
\forall v,w \in V: (\exists \lambda\in\mathbb{R} v = \lambda w) \Rightarrow \vert\langle v,w\rangle\vert = \Vert v\Vert\cdot \Vert w\Vert
\]
\end{enumerate}

\subsubsection*{Bewijs}
\begin{proof}
Bewijs door gevalsonderscheid.\\
Als $v$ en $w$ nulvectoren zijn gelden beide beweringen.
\[
\vert\langle \vec{0},\vec{0}\rangle\vert =0= \Vert \vec{0}\Vert\cdot \Vert \vec{0}\Vert
\]
We bewijzen de beweringen nu nog voor de andere gevallen.

\begin{enumerate}
\item
We weten dat voor willekeurige $v,w\in V$ geldt dat $\langle v + \lambda w,v+\lambda w\rangle \ge 0$. We werken dit uit.
\[
\langle v + \lambda w,v+\lambda w\rangle = \langle v,v+\lambda w \rangle + \lambda \langle w,v+\lambda w\rangle = \langle v,v \rangle + \lambda \langle v,w \rangle + \lambda \langle w,v\rangle + \lambda^2\langle w,w\rangle
\]
\[
\langle v,v \rangle + 2\lambda \langle v,w \rangle + \lambda^2\langle w,w\rangle \ge 0
\]
Het linker lid van deze ongelijkheid is een tweedegraads veelterm waarvan we weten dat ze altijd positief is. Bijgevolg is de discriminant ervan negatief.
\[
4\langle v,w\rangle^2-4\langle v,v\rangle\langle w,w\rangle \le 0
\]
\[
\langle v,w\rangle^2-\langle v,v\rangle\langle w,w\rangle \le 0
\]
\[
\langle v,w\rangle^2 \le \langle v,v\rangle\langle w,w\rangle 
\]
Voor diegenen die dit gefoefel vinden, u heeft gelijk.

\item
Als $w$ en $v$ lineair afhankelijk zijn dan bestaat er dus een $\lambda\in\mathbb{R}$ zodat $v=\lambda w$.
\[
|\langle \lambda w, w \rangle| = |\lambda  \langle w, w \rangle| = |\lambda\sqrt{\langle w, w \rangle^2}| = |\sqrt{\lambda^2\langle w, w \rangle^2}| = |\sqrt{\langle \lambda w, \lambda w \rangle \cdot \langle w, w \rangle }| \]
\[
= |\sqrt{\langle \lambda w, \lambda w \rangle} \sqrt{\langle w, w \rangle }| = |\sqrt{\langle v, v \rangle} \sqrt{\langle w, w \rangle }| = \Vert v\Vert\cdot \Vert w\Vert
\]
\end{enumerate}
\end{proof}

\subsection{Definitie 6.16 p 231}
Zij $(\mathbb{R},V,+,\langle\cdot,\cdot\rangle)$ een inproductruimte.
\subsubsection*{Te Bewijzen}
Voor elke $v,w\in V$ bestaat er een unieke hoek $\theta$ zodat de volgende gelijkheid geldt.
\[
\cos\theta = \frac{\langle v,w\rangle}{\Vert v\Vert\cdot \Vert w\Vert}
\]
\subsubsection*{Bewijs}
\begin{proof}
Direct bewijs\\
We weten dat voor elke $v,w\in V$ het volgende geldt.
\[
\vert\langle v,w\rangle\vert \le \Vert v\Vert\cdot \Vert w\Vert
\]
Hieruit volgt het volgende, zorg dat je dit begrijpt.
\[
-1 \le \frac{\langle v,w\rangle}{\Vert v\Vert\cdot \Vert w\Vert} \le 1
\]
We weten dat voor elk getal tussen $-1$ en $1$ er een hoek bestaat zodat de cosinus ervan dat getal is.

\end{proof}



\section{Oefeningen 6.8}






\end{document}