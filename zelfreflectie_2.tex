\documentclass[lineaire_algebra_oplossingen.tex]{subfiles}
\begin{document}

\part{Zelfreflectie 2}
\section{oef 1}
\[
C_{11} = 
\begin{vmatrix}
5 & 6 \\
8 & 9
\end{vmatrix}
=-3
\]
\[
C_{22}= 
\begin{vmatrix}
1 & 3 \\
7 & 9
\end{vmatrix}
=
18
\]
\[
C_{12} =
\begin{vmatrix}
4 & 6 \\
7 & 9
\end{vmatrix}
=-6
\]
\[
M_{11} = 
\begin{pmatrix}
5 & 6 \\
8 & 9
\end{pmatrix}
\]
\[
M_{22}= 
\begin{pmatrix}
1 & 3 \\
7 & 9
\end{pmatrix}
\]
\[
M_{12} =
\begin{pmatrix}
4 & 6 \\
7 & 9
\end{pmatrix}
\]
\section{oef 2}
\subsection*{Te bewijzen}
$A$ is inverteerbaar $\Rightarrow$ $f(A^{-1}) = f(A)^{-1}$
\subsection*{Bewijs}
"$A$ is inverteerbaar" betekent het volgende.
\[
\exists B: A\cdot B = I_n = B\cdot A
\]
Noteer $B = A^{-1}$ 
\begin{proof}
Rechtstreeks bewijs.\\
Noteer de inverse van $A$ als $B$.
\[
f(B) = f(A)^{-1}
\]
\[
f(B)\cdot f(A) = f(A)^{-1}f(A)
\]
Definitie determinant afbeelding
\[
f(B\cdot A) = 1
\]
$A$ is inverteerbaar met $B$ als inverse.
\[
f(I_n) = 1
\]
\[
1 = 1
\]
\[
True
\]


\end{proof}
\section{oef 3}
\section{oef 4}
\section{oef 5}
\section{oef 6}
\section{oef 7}
\end{document}