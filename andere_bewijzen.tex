\documentclass[lineaire_algebra_oplossingen.tex]{subfiles}
\begin{document}

\chapter{Andere Bewijzen}
\section{Matrices}

\subsection{Scheefsymmetrische Matrix}
\begin{itemize}
\item
Zij $A$ een scheefsymmetrische matrix.
\subsubsection*{Te Bewijzen}
De elementen op de hoofddiagonaal van $A$ zijn allemaal nul.

\subsubsection*{Bewijs}
\begin{proof}
$A$ is scheefsymmetrisch. We beschouwen nu een element op de hoofddiagonaal van $A$. Noem $a$ het $i$-de element op de hoofddiagonaal $(A)_{ii}$.
\[
(-A)_{ii} = (A^T)_{ii}
\]
\[
-a = a
\]
\[
a = 0
\]
\end{proof}
\end{itemize}

\section{Rijruimte, Nulruimte en Kolomruimte}
\subsection*{Orthogonaal Complement}
\begin{itemize}
\item Zij $A$ een $\mathbb{R}^{m \times n}$. Beschouw $A$ als een kolom van rijen $a_i$.
\[
A = 
\begin{pmatrix}
a_1\\a_2\\\vdots\\a_m
\end{pmatrix}
\] 

\subsubsection*{Te Bewijzen}
De rijruimte van $A$ is het orthogonaal complement van de nulruimte van $A$.

\subsubsection*{Bewijs}
\begin{proof}
De nulruimte van $A$ bestaat uit vectoren $x$ zodat $Ax = \vec{0}$ geldt.
\[
A \cdot x = 
\begin{pmatrix}
a_1 x\\a_2 x\\\vdots\\a_m x  
\end{pmatrix}
=
\vec{0}
\]
Dit betekent dat het inproduct van elke rij van $A$ met $x$ nul is. Dit houdt precies in dat de rijruimte van $A$ het orthogonaal complement is van de nulruimte van $A$.
\end{proof}



\end{itemize}


\end{document}